\chapter{Related Work}\label{related_work}

In this chapter, a literature review is presented, focusing on the topics of transformers, learning arithmetic tasks, reasoning capabilities, compositional learning, and generalization to longer sequence lengths.

\TODO{rewrite summary}

\section{Learning Arithmetics with Transformers}\label{sec:sota_arithmetic_tasks}

The ability of transformer models \parencite{vaswani_attention_2017} to learn arithmetic operations such as integer addition has been a subject of significant research interest. Evaluations demonstrated that large pre-trained language models such as GPT-3 \parencite{brown_language_2020} can exhibit emergent capabilities across general-purpose tasks, including basic few-digit arithmetic, despite these tasks not being explicitly encoded by the unsupervised next-token prediction objective. However, even largest state-of-the-art models like GPT-4 \parencite{achiam_gpt-4_2023} struggle to robustly solve multi-digit addition and multiplication, especially when a larger number of digits are involved.

\cite{lee_teaching_2023} investigate how even small transformers, trained from random initialization, can efficiently learn arithmetic operations such as addition and multiplication. They show that training on chain-of-thought style data that includes intermediate step results significantly improves accuracy, sample complexity, and convergence speed, even in the absence of pretraining. This approach aligns with the experiments presented in this thesis on chain-of-thought training, where models are trained to output the steps involved in solving addition problems. One limitation of their work is that it limits each task to a fixed number of digits (e.g. 7 and 7 digit operands), using padding to ensure uniform input length. In contrast, this thesis extends the task to variable-length addition problems which is more difficult due to the need for models to learn to position-wise align digits as discussed in Chapter \ref{introduction}.

Understanding how transformers learn arithmetic tasks is further explored by \cite{quirke_understanding_2023}, who present an in-depth mechanistic analysis of a one-layer transformer model trained for integer addition. They reveal that the model processes the problem in a non-intuitive way: it divides the task into parallel, digit-specific streams and employs distinct algorithms for different digit positions, merging the results in the MLP layer. This work also restricts the operands to a fixed length of 5 digits and employs padding, apart from restricting the model to a single layer.

Length generalization is a critical challenge in training transformers for arithmetic tasks that comes up in numerous research works. \cite{jelassi_length_2023} examine how transformers cope with learning basic integer arithmetic and generalizing to longer sequences than seen during training. They find that relative position embeddings enable length generalization for simple tasks such as addition, allowing models trained on 5-digit numbers to perform 15-digit sums. However, this method fails for multiplication, leading them to propose ``train set priming'' by adding a few (10 to 50) longer sequences to the training set. Despite showing interesting capabilites of learning from few examples, this defeats the purpose of \emph{zero-shot} generalization to unseen lengths.

Similarly, \cite{duan_interpolation_2023} investigate the capabilities of transformers in learning arithmetic algorithms and introduce Attention Bias Calibration (ABC), a calibration stage that enables the model to automatically learn proper attention biases linked to relative position encoding mechanisms. Using ABC, they achieve robust length generalization on certain arithmetic tasks. Despite promising results, this work is limited due to the attention bias intervention being task-specific and the need for a modified training with 2 stages (first training to perfect interpolation accuracy to learn the attention biases, then training another model with extracted attention biases). Conversely, the main research interest  in this domain lies in architectural modifications that apart from boosting algorithmic capabilities, also preserve or improve performance on other language tasks.

Recent work by \cite{mcleish_transformers_2024} addresses the poor performance of transformers on arithmetic tasks by adding an embedding to each digit that encodes its position relative to the start of the number. This modification, along with architectural changes like input injection and recurrent layers, significantly improves performance, achieving up to 99\% accuracy on 100-digit addition problems. As of now, this work represents the state-of-the-art in length generalization on integer addition, with test sequence lengths up to 6 times longer training sequence lengths (compared to previous SOTA of 2.5 times by \cite{zhou_transformers_2024}).

Investigations into the symbolic capabilities of large language models by \cite{dave_investigating_2024} and the arithmetic properties in the space of language model prompts by \cite{krubinski_basic_2023} further contribute to understanding how transformers process arithmetic operations and the challenges involved in symbolic reasoning tasks. Unlike this work and other studies that use smaller transformer models trained from scratch, these works focus on large pre-trained models which exhibit emergent capabilities in arithmetic tasks through large-scale unsupervised training.

Mechanistic interpretability of transformers on arithmetic tasks is further explored by \cite{nanda_progress_2022}, who study the phenomenon of grokking in small transformer models trained on modular addition tasks. They fully reverse-engineer the learned algorithm and discover that the model implements a discrete Fourier transform and uses trigonometric identities to convert addition into rotations on a circle. Similarly, \cite{zhong_clock_2023} investigate the mechanistic explanations of neural networks on modular addition tasks. They find that neural networks can discover multiple qualitatively different algorithms when trained on the same task, including known algorithms like the ``Clock'' algorithm (same as \cite{nanda_progress_2022}) and a novel ``Pizza'' algorithm. Unlike the focus of this thesis, which deals with integer addition where each digit is treated as a separate token, these works are limited to modular addition tasks where each number is a single token, i.e. \emph{123} token instead of tokens for \emph{1}, \emph{2}, and \emph{3}. This simplification allows for a detiled mechanistic understanding of the model but does not address the challenges associated with variable-length inputs and position-wise alignment of digits in full integer addition. Moreover, their primary concern is interpretability and understanding training dynamics, rather than improving length generalization or exploring failure modes in more complex arithmetic tasks.

In summary, the literature demonstrates that transformers can learn arithmetic tasks like integer addition, but challenges remain in achieving robust length generalization and understanding the underlying mechanisms by which these models perform arithmetic operations. These findings are directly relevant to the focus of this thesis, i. e. length generalization on integer addition and failure modes of transformers, since works either simplify the task to be modular addition (\cite{nanda_progress_2022}, \cite{zhong_clock_2023}), fixed digit length (\cite{lee_teaching_2023}, \cite{quirke_understanding_2023}), or propose task-specific architectural modifications to improve performance (\cite{mcleish_transformers_2024}) whose performance on other language tasks is not explored.

\section{Reasoning in Transformers}\label{sec:sota_reasoning_in_transformers}

Transformers have shown remarkable abilities in reasoning tasks, particularly when employing techniques such as including a chain-of-thought (CoT) in the sequence instead of directly outputting an answer. \cite{li_chain_2024} provide a theoretical understanding of the power of chain-of-thought for decoder-only transformers, demonstrating that CoT empowers the model with the ability to perform inherently serial computation, which is otherwise lacking in transformers, especially for fewer layers. Intuitively, this is due to the fact that a CoT part in generated sequence allows the model to write out intermediate results and perform more computation before arriving at the final answer. This theoretical perspective also supports the experiments described in Chapter \ref{approach} involving chain-of-thought training to enhance the reasoning capabilities of models on arithmetic tasks.

\cite{wang_cot_2024} explore the idea that chain-of-thought reasoning paths can be elicited from pre-trained language models by simply altering the decoding process, rather than relying on specific prompting techniques. Instead of decoding by taking the tokens with most activation (greedy decoding), they propose evaluating multiple possible first tokens, and then continue with greedy decoding for each of them. This results in multiple decoded sequences instead of a single answer sequence. They find that paths including a chain-of-thought part already frequently exist among these alternative sequences and that the presence of a CoT in the decoding path correlates with higher confidence in the model's decoded answer. This work strengthens the argument that chain-of-thought reasoning is a powerful mechanism for transformers to improve reasoning capabilities.

Another research direction is training the models to output a chain-of-thought sequence using bootstrapping from existing data and pre-trained models, without the need for curated CoT data. \cite{zelikman_star_2022} propose the Self-Taught Reasoner (STaR) method to bootstrap reasoning capabilities using existing models and answer-only datasets. In STaR, a pre-trained LLM is encouraged to generate intermediate steps before answer using few-shot prompting, and it is assumed that if the resulting answer is correct, the generated CoT steps are also correct. Then, the model is fine-tuned on the generated CoT data. This method is shown to improve reasoning capabilities on various tasks, including arithmetic. Further extending the concept of self-generated reasoning, \cite{zelikman_quiet-star_2024} introduce Quiet-STaR, where language models learn to generate ``rationales'' at each token to explain future text using reinforcement learning, thereby improving their predictions. This approach generalizes previous work on self-taught reasoning by \cite{zelikman_star_2022}, training the LLM to implicitly reason without explicit generation of a CoT trace, and can leverage unsupervised text datasets since the objective is not task-specific.

\cite{goyal_think_2024} propose training language models with ``pause tokens'', allowing the model to perform more computation before outputting the next tokens. This method improves performance on reasoning tasks, including arithmetic. However, while it is tempting to think that the pause tokens implicitly perform a form of chain-of-thought reasoning, the authors do not explicitly analyze the internal reasoning processes of the model. Moreover, the pause tokens are not directly interpretable as distinct, structured intermediate steps, which is a key feature of chain-of-thought reasoning.

The limitations of transformers in performing counting tasks are highlighted by \cite{yehudai_when_2024}, who focus on simple counting tasks involving counting the number of times a token appears in a string. They show that transformers can solve this task under certain conditions, such as when the dimension of the transformer state is linear in the context length. But in general this ability does not scale beyond this limit, based on the authors' theoretical arguments for the observed limitations. While this work does not directly address arithmetic tasks, it provides insights into the limitations of transformers in handling a related counting task.

Understanding how transformers learn causal structures is investigated by \cite{nichani_how_2024}, who introduce an in-context learning task requiring the learning of latent causal structures. They prove that gradient descent on a simplified two-layer transformer learns to solve this task by encoding the latent causal graph in the first attention layer. However, this work is highly theoretical and does not directly address arithmetic tasks, nor other algorithmic or language tasks.

Overall, despite some theoretical results proving possibility of certain forms of reasoning, the literature does not show a robust solution that allows transformers to learn the generalizable algorithm for performing integer addition on an unbounded number of digits, which is the main focus of this thesis. Moreover, even if theoretical possibility was to be established, it would still be of interest to interpret the learned algorithm in a higher-level, human-understandable form akin to \cite{nanda_progress_2022}.

\section{Compositional Learning}\label{sec:sota_compositional_learning}

Compositionality refers to the ability of models to understand and generate new combinations of known components.
\TODO{cite \cite{hupkes_compositionality_2020} and expand definition}

The work of \cite{hupkes_compositionality_2020} provides a theoretical framework for understanding compositional generalization in neural networks. They propose tests to investigate whether models systematically recombine known parts and rules, generalize to longer sequences, and favor rules over exceptions during training. This framework is relevant to our analysis of how transformers learn and generalize in arithmetic tasks.

\cite{dziri_faith_2023} investigate the limits of transformers on compositional tasks, such as multi-digit multiplication and logic grid puzzles. They find that transformers tend to solve compositional tasks by reducing multi-step reasoning into \emph{linearized subgraph matching rather} than developing systematic problem-solving skills, suggesting limitations in compositional generalization. Apart from suggesting an interesting description for how multi-step reasoning is realized on transformers, this work does not explore integer addition problems, instead focusing on <5 digit multiplication and logic puzzles.

\cite{press_measuring_2023} measure the compositionality gap in language models by evaluating how often models can correctly answer all sub-problems but fail to generate the overall solution. They find that as model size increases, single-hop question answering performance improves faster than multi-hop performance, indicating that while larger models have better factual recall, they do not necessarily improve in their ability to perform compositional reasoning. They propose methods like chain-of-thought and self-ask prompting to narrow the compositionality gap.

\cite{zhou_what_2023} propose the RASP-Generalization Conjecture, suggesting that transformers tend to length generalize on a task if it can be solved by a short program (in a domain-specific language describing the transformer architecture) that works for all input lengths. They use this framework to understand when and how transformers exhibit strong length generalization on algorithmic tasks, which is closely related to compositional learning.

In summary, while transformers have demonstrated some ability to perform compositional tasks, significant challenges remain in achieving systematic compositional generalization. These findings inform this thesis' focus on multi-task learning and understanding how models can be trained to perform compositional operations like digit alignment and modular sum, before correctly combining results from these sub-tasks into a final answer.

\section{Length Generalization}\label{sec:sota_length_generalization}

Generalization to longer sequence lengths is a critical challenge in training transformers for tasks like integer addition. Positional encoding plays a significant role in length generalization. \cite{kazemnejad_impact_2023} conduct a systematic empirical study comparing different positional encoding approaches, including Absolute Position Embedding (APE) \parencite{vaswani_attention_2017}, Relative position encoding \parencite{shaw_self-attention_2018}, ALiBi \parencite{press_train_2021}, Rotary osition encoding (RoPE) \parencite{su_roformer_2024}, and Transformers without positional encoding (NoPE). Interestingly, they find that explicit position embeddings are not essential for decoder-only transformers to generalize to longer sequences and that models without positional encoding outperform others in length generalization. The results of experiments conducted in this thesis are not consistent with these findings, observing much steeper performance drops on longer sequences when using NoPE, RoPE or APE.

\cite{ruoss_randomized_2023} introduce randomized positional encodings to boost length generalization of transformers. They demonstrate that their method allows transformers to generalize to sequences of unseen length by simulating the positions of longer sequences and randomly selecting an ordered subset from this longer sequence. Experiments in this thesis also confirm that randomized positional encodings can improve length generalization on integer addition tasks. Moreover, experiments show that even without architectural changes, simply randomly adding spaces to the input sequence can simulate randomized positions and boost out-of-distribution accuracy for task lengths marginally longer than those encountered during training. The latter claim is also supported by work of \cite{shen_positional_2023}.

\cite{li_functional_2024} propose a novel functional relative position encoding with progressive interpolation (FIRE) to improve transformer generalization to longer contexts. They theoretically show that FIRE can learn to represent other relative position encodings and demonstrate empirically that FIRE positional encoding enables models to better generalize to longer contexts on long text benchmarks.

The success of length generalization is also linked to data format and position encoding type. \cite{zhou_transformers_2024} test the transformer's ability to generalize using the two integer addition task. They show that transformers can achieve limited length generalization, but find that the performance depends on random weight initialization as well. Their approach uses FIRE positional encoding and index hints (e.g. ``123+456=579'' is encoded as ``a1b2c3+a4b5c6=a5b7c9''), achieving 2.5x length generalization ratio (maximum length of successfully predicted test sequences to training sequences). This work is related to the experiments conducted in this thesis, which also explore the impact of different positional encodings and data formatting on length generalization in integer addition tasks. Before this work, the state-of-the-art length generalization ratio was 1.5x \parencite{zhou_what_2023}, and before that, 1.124x \parencite{kazemnejad_impact_2023} and 1.1x \parencite{shen_positional_2023}.

As mentioned in Section \ref{sec:sota_arithmetic_tasks}, \cite{mcleish_transformers_2024} address length generalization by adding an embedding to each digit that encodes its position relative to the start of the number. This fix, along with architectural modifications such as input injection and recurrent application of layers, allows transformers to generalize to larger and more complex arithmetic problems. Abacus embeddings generalize to 120-digit problems when trained on up to 20-digit problems, achieving state-of-the-art performance of 6x length generalization ratio on integer addition tasks.

\bigskip

In the experiments presented in this thesis, different positional encodings and multi-task training are explored to improve length generalization in integer addition tasks. The literature suggests that carefully designed positional encodings and data formatting can significantly impact the ability of transformers to generalize to longer sequences. Moreover, this work isolates the causes of failure in length generalization on integer addition, and proposes minimally invasive modifications to improve performance on longer sequences.