\documentclass[14pt,aspectratio=169]{beamer}
%%%%%%%%%%%%%%%%%%%%%%%%%%%%%%%%%%%%%%%%%%%%%%%%%%%%%%%%%%%%%
% Meta informations:
\newcommand{\trauthor}{Imran Ibrahimli}
\newcommand{\trcourse}{Intelligent Adaptive Systems}
\newcommand{\trtitle}{Length Generalization on Multi-Digit Integer Addition with Transformers}
\newcommand{\trmatrikelnummer}{7486484}
\newcommand{\tremail}{imran.ibrahimli@studium.uni-hamburg.de}
\newcommand{\trinstitute}{Dept. Informatik -- Knowledge Technology, WTM}
\newcommand{\trwebsiteordate}{{https://www.informatik.uni-hamburg.de/WTM/}}

%%%%%%%%%%%%%%%%%%%%%%%%%%%%%%%%%%%%%%%%%%%%%%%%%%%%%%%%%%%%%
% Languages:
\usepackage[english]{babel}
\selectlanguage{english}

%%%%%%%%%%%%%%%%%%%%%%%%%%%%%%%%%%%%%%%%%%%%%%%%%%%%%%%%%%%%%
% Bind packages:
\usepackage{beamerthemesplit}
\usetheme{Boadilla}
%\usetheme{Copenhagen}
%\usetheme{Darmstadt}
%\usetheme{Frankfurt}
%\usetheme{Ilmenau}
%\usetheme{JuanLesPins}
%\usetheme{Madrid}
%\usetheme{Warsaw }
%\usecolortheme{dolphin}
%\setbeamertemplate{sections/subsections in toc}[sections numbered]
%\beamertemplatenavigationsymbolsempty
%\setbeamertemplate{headline}[default] 	% deaktiviert die Kopfzeile
\setbeamertemplate{navigation symbols}{}% deaktiviert Navigationssymbole
%\useinnertheme{rounded}

\usepackage{acronym}                    % Acronyms
\usepackage{algorithmic}								% Algorithms and Pseudocode
\usepackage{algorithm}									% Algorithms and Pseudocode
\usepackage{amsfonts}                   % AMS Math Packet (Fonts)
\usepackage{amsmath}                    % AMS Math Packet
\usepackage{amssymb}                    % Additional mathematical symbols
\usepackage{amsthm}
\usepackage{color}                      % Enables defining of colors via \definecolor
\usepackage{fancybox}                   % Gleichungen einrahmen
\usepackage{fancyhdr}										% Paket zur schickeren der Gestaltung der 
\usepackage{graphicx}                   % Inclusion of graphics
%\usepackage{latexsym}                  % Special symbols
\usepackage{longtable}									% Allow tables over several parges
\usepackage{listings}                   % Nicer source code listings
\usepackage{lmodern}
\usepackage{multicol}										% Content of a table over several columns
\usepackage{multirow}										% Content of a table over several rows
\usepackage{rotating}										% Alows to rotate text and objects
\usepackage[section]{placeins}          % Ermoeglich \Floatbarrier fuer Gleitobj. 
\usepackage[hang]{subfigure}            % Allows to use multiple (partial) figures in a fig
%\usepackage[font=footnotesize,labelfont=rm]{subfig}	% Pictures in a floating environment
\usepackage{tabularx}										% Tables with fixed width but variable rows
\usepackage{url,xspace,boxedminipage}   % Accurate display of URLs


\definecolor{uhhRed}{RGB}{226,0,26}     % Official Uni Hamburg Red
\definecolor{uhhGrey}{RGB}{136,136,136} % Official Uni Hamburg Grey
\definecolor{uhhLightGrey}{RGB}{220, 220, 220}
\setbeamertemplate{itemize items}[ball]
\setbeamercolor{title}{fg=uhhRed,bg=white}
\setbeamercolor{title in head/foot}{bg=uhhRed}
\setbeamercolor{block title}{bg=uhhGrey,fg=white}
\setbeamercolor{block body}{bg=uhhLightGrey,fg=black}
\setbeamercolor{section in head/foot}{bg=black}
\setbeamercolor{frametitle}{bg=white,fg=uhhRed}
\setbeamercolor{author in head/foot}{bg=black,fg=white}
\setbeamercolor{author in footline}{bg=white,fg=black}
\setbeamercolor*{item}{fg=uhhRed}
\setbeamercolor*{section in toc}{fg=black}
\setbeamercolor*{separation line}{bg=black}
\setbeamerfont*{author in footline}{size=\scriptsize,series=\mdseries}
\setbeamerfont*{institute}{size=\footnotesize}

\newcommand{\opticalseperator}{0.0025\paperwidth}

\institute{Universit\"at Hamburg\\\trinstitute}
\title{\trtitle}
\author{\trauthor}
\date{}
\logo{}

%%%%%%%%%%%%%%%%%%%%%%%%%%%%%%%%%%%%%%%%%%%%%%%%%%%%%%%%%%%%%
% Configurationen:
%\hypersetup{pdfpagemode=FullScreen}

\hyphenation{whe-ther} 									% Manually use: "\-" in a word: Staats\-ver\-trag

%\lstloadlanguages{C}                   % Set the default language for listings
\DeclareGraphicsExtensions{.pdf,.svg,.jpg,.png,.eps} % first try pdf, then eps, png and jpg
\graphicspath{{./img/}} 								% Path to a folder where all pictures are located

%%%%%%%%%%%%%%%%%%%%%%%%%%%%
% Costom Definitions:
\setbeamertemplate{title page}
{
  \vbox{}
	\vspace{0.4cm}
  \begin{centering}
    \begin{beamercolorbox}[sep=8pt,center,colsep=-4bp]{title}
      \usebeamerfont{title}\inserttitle\par%
      \ifx\insertsubtitle\@empty%
      \else%
        \vskip0.20em%
        {\usebeamerfont{subtitle}\usebeamercolor[fg]{subtitle}\insertsubtitle\par}%
      \fi%     
    \end{beamercolorbox}%
		\vskip0.4em
    \begin{beamercolorbox}[sep=8pt,center,colsep=-4bp,rounded=true,shadow=true]{author}
      \usebeamerfont{author}\insertauthor \\ \insertinstitute
    \end{beamercolorbox}

	  \vfill
	  \begin{beamercolorbox}[ht=10ex,center]{}
		  \includegraphics[width=0.175\paperwidth]{wtmIcon.pdf}
	  \end{beamercolorbox}%
    \begin{beamercolorbox}[sep=8pt,center,colsep=-4bp,rounded=true,shadow=true]{institute}
      \usebeamerfont{institute}\trwebsiteordate
    \end{beamercolorbox}
		\vspace{-0.1cm}
  \end{centering}
}

\setbeamertemplate{frametitle}
{
\begin{beamercolorbox}[wd=\paperwidth,ht=3.8ex,dp=1.2ex,leftskip=0pt,rightskip=4.0ex]{frametitle}%
		\usebeamerfont*{frametitle}\centerline{\insertframetitle}
	\end{beamercolorbox}
	\vspace{0.0cm}
}

\setbeamertemplate{footline}
{
  \leavevmode
	\vspace{-0.05cm}
  \hbox{
	  \begin{beamercolorbox}[wd=.32\paperwidth,ht=4.8ex,dp=2.7ex,center]{author in footline}
	    \hspace*{2ex}\usebeamerfont*{author in footline}\trauthor
	  \end{beamercolorbox}%
	  \begin{beamercolorbox}[wd=.60\paperwidth,ht=4.8ex,dp=2.7ex,center]{author in footline}
	    \usebeamerfont*{author in footline}\trtitle
	  \end{beamercolorbox}%
	  \begin{beamercolorbox}[wd=.07\paperwidth,ht=4.8ex,dp=2.7ex,center]{author in footline}
	    \usebeamerfont*{author in footline}\insertframenumber{}
	  \end{beamercolorbox}
  }
	\vspace{0.15cm}
}
\renewcommand{\footnotesize}{\fontsize{12.4pt}{12.4pt}\selectfont}
\renewcommand{\small}{\fontsize{13.8pt}{13.8pt}\selectfont}
\renewcommand{\normalsize}{\fontsize{15.15pt}{15.15pt}\selectfont}
\renewcommand{\large}{\fontsize{17.7pt}{17.7pt}\selectfont}
\renewcommand{\Large}{\fontsize{21.3pt}{21.3pt}\selectfont}

%%%%%%%%%%%%%%%%%%%%%%%%%%%%
% Additional 'theorem' and 'definition' blocks:
\newtheorem{axiom}{Axiom}[section] 	
%\newtheorem{axiom}{Fakt}[section]			% Wenn in Deutsch geschrieben wird.
%Usage:%\begin{axiom}[optional description]%Main part%\end{fakt}

%Additional types of axioms:
\newtheorem{observation}[axiom]{Observation}

%Additional types of definitions:
\theoremstyle{remark}
%\newtheorem{remark}[section]{Bemerkung} % Wenn in Deutsch geschrieben wird.
\newtheorem{remark}[section]{Remark} 

%%%%%%%%%%%%%%%%%%%%%%%%%%%%
% Provides TODOs within the margin:
\newcommand{\TODO}[1]{\marginpar{\emph{\small{{\bf TODO: } #1}}}}

%%%%%%%%%%%%%%%%%%%%%%%%%%%%
% Abbreviations and mathematical symbols
\newcommand{\modd}{\text{ mod }}
\newcommand{\RS}{\mathbb{R}}
\newcommand{\NS}{\mathbb{N}}
\newcommand{\ZS}{\mathbb{Z}}
\newcommand{\dnormal}{\mathit{N}}
\newcommand{\duniform}{\mathit{U}}

\newcommand{\erdos}{Erd\H{o}s}
\newcommand{\renyi}{-R\'{e}nyi}

%%%%%%%%%%%%%%%%%%%%%%%%%%%%
% Display of TOCs:
\AtBeginSection[]
{
	\setcounter{tocdepth}{2}  
	\frame
	{
	  \frametitle{Outline}
		\tableofcontents[currentsection]
	}
}
 

%%%%%%%%%%%%%%%%%%%%%%%%%%%%%%%%%%%%%%%%%%%%%%%%%%%%%%%%%%%%%
% Document:
\begin{document}

\begin{frame}[plain]
    \titlepage
\end{frame}

\frame{
    \frametitle{Outline}
    \tableofcontents
}

\section{Introduction}

\begin{frame}
    \frametitle{Introduction}
    \begin{itemize}
        \item Overview of length generalization challenges in sequence tasks
        \item Importance of transformers for AI applications
    \end{itemize}
\end{frame}

\section{Motivation}

\begin{frame}
    \frametitle{Motivation}
    \begin{itemize}
        \item Challenges with transformers in multi-digit addition
        \item Real-world implications of length generalization issues
    \end{itemize}
\end{frame}

\begin{frame}
    \frametitle{Problem Statement}
    \begin{itemize}
        \item Transformers struggle with sequences longer than seen in training
        \item Importance of positional encoding for accurate digit alignment
    \end{itemize}
\end{frame}

\section{Research Questions}

\begin{frame}
    \frametitle{Research Questions}
    \begin{itemize}
        \item \textbf{RQ1}: Why do transformers with standard positional encodings fail in length generalization?
        \item \textbf{RQ2}: How do sub-task data influence length generalization?
        \item \textbf{RQ3}: How can interpretability techniques reveal transformer mechanisms?
    \end{itemize}
\end{frame}

\section{Background}

\begin{frame}
    \frametitle{Background - Transformers}
    \begin{itemize}
        \item Key transformer components: self-attention, feed-forward networks
        \item Encoder-decoder vs. decoder-only models
    \end{itemize}
\end{frame}

\begin{frame}
    \frametitle{Background - Positional Encoding}
    \begin{itemize}
        \item Absolute, relative, and random encoding methods
        \item Importance of positional encoding in handling sequence tasks
    \end{itemize}
\end{frame}

\begin{frame}
    \frametitle{Related Work}
    \begin{itemize}
        \item Overview of research on length generalization and transformers in arithmetic tasks
    \end{itemize}
\end{frame}

\section{Approach}

\begin{frame}
    \frametitle{Approach - Overview}
    \begin{itemize}
        \item Focus on minimal architectural changes and positional encoding
    \end{itemize}
\end{frame}

\begin{frame}
    \frametitle{Experimental Setup}
    \begin{itemize}
        \item Model configurations and evaluation metrics
    \end{itemize}
\end{frame}

\begin{frame}
    \frametitle{Data Formatting}
    \begin{itemize}
        \item Techniques: zero padding, reversing answer, etc.
    \end{itemize}
\end{frame}

\begin{frame}
    \frametitle{Limitations of Absolute Positional Encoding}
    \begin{itemize}
        \item Issues with absolute encoding and digit alignment
    \end{itemize}
\end{frame}

\begin{frame}
    \frametitle{Random Spaces Technique}
    \begin{itemize}
        \item Description and impact on model generalization
    \end{itemize}
\end{frame}

\begin{frame}
    \frametitle{Results - Baseline Model Performance}
    \begin{itemize}
        \item Baseline results with standard positional encoding
    \end{itemize}
\end{frame}

\begin{frame}
    \frametitle{Attention Map Analysis}
    \begin{itemize}
        \item Visual analysis of attention patterns
    \end{itemize}
\end{frame}

\begin{frame}
    \frametitle{Impact of Data Formatting}
    \begin{itemize}
        \item Comparing results across different formatting methods
    \end{itemize}
\end{frame}

\begin{frame}
    \frametitle{Sub-task Learning}
    \begin{itemize}
        \item Role of sub-tasks in improving length generalization
    \end{itemize}
\end{frame}

\begin{frame}
    \frametitle{Discussion - Length Generalization}
    \begin{itemize}
        \item Key findings on generalization capabilities
    \end{itemize}
\end{frame}

\begin{frame}
    \frametitle{Challenges and Limitations}
    \begin{itemize}
        \item Limitations of random spaces and Abacus encoding
    \end{itemize}
\end{frame}

\section{Conclusion}

\begin{frame}
    \frametitle{Conclusion}
    \begin{itemize}
        \item Summary of findings and contributions
    \end{itemize}
\end{frame}

\begin{frame}
    \frametitle{Future Work}
    \begin{itemize}
        \item Suggestions for further research in encoding methods and model interpretability
    \end{itemize}
\end{frame}

\begin{frame}
    \frametitle{The End}
    \begin{center}
        Thank you for your attention. \\[1ex]
        Any questions?
    \end{center}
\end{frame}

\end{document}
